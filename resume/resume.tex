\documentclass[a4paper]{article}
\usepackage{array}
\usepackage{setspace}

\addtolength{\oddsidemargin}{-.875in}
\addtolength{\evensidemargin}{-.875in}
\addtolength{\topmargin}{-1.5in}

\addtolength{\textwidth}{1.75in}

\begin{document}
    \pagenumbering{gobble}
    \begin{center}
        \begin{LARGE}
            Alex Laties \\
        \end{LARGE}
        387 Franklin Avenue, Apt 103\\
        Brooklyn, NY 11238\\
        Phone: 267 825 2530\\
        Website: http://alex.laties.info/ \\
        E-Mail: alex@laties.info \\
    \end{center}

    \vspace{.2in}
    \begin{center}
    \begin{Large}
        General Purpose Software Developer
    \end{Large}
    \end{center}

    \vspace{.2in}
    \begin{large}
    Alex Laties\textsuperscript{\texttrademark} is a general purpose programmer that has worked on building, maintaining, and scaling web applications. He primarily works in Python, but has no strong language preference and has demonstrated some faculty with C, C\#, Java, Ruby, PHP, and Javascript.  He has worked in larger IDEs, such as Visual Studio, MonoDevelop, and XCode, but claims no great proficiencies in any of them. He primarily uses some hodge podge of Vim and assorted plugins as an IDE. \\
    \par
    \indent Alex has worked with teams in an agile style to quickly deliver features and products. He has worked with ticketing systems, such as Jira and Trello, to manage work loads. He has worked with source control systems, like Subversion and Git, to prevent himself and teammates from going crazy. He has worked in a test driven style and found it rather nice.  \\
    \par
    \indent Alex has contributed some code to open source projects Bottle and pyrollbar.
    \end{large}
    \pagebreak

    \begin{Large}
    Work History \\
    \end{Large}
    \begin{tabular}{ m{3cm} m{3cm} m{3cm} m{8cm} }
        \hline
        March 2012 - \newline October 2013 & Voxy, Inc & Software Engineer & 
            \begin{itemize}
                \item Worked with Django 1.3 on Python 2.7 to maintain a web application and mobile API.
                \item Built prototypes for internal CRMs.
                \item Built a new mobile API stack in Bottle.
                \item Set up loadbalancing for web and mobile stacks using haproxy and pacemaker.
                \item Set up and wrote initial recipes for Chef provisioning service.
                \item Also dealt with PostgreSQL, MongoDB, Varnish, Celery, Redis, uwsgi, gunicorn, and nginx. 
            \end{itemize}
        \\
        \hline
        Fall/Winter 2010 - January 2011 & UPenn \newline School of Medicine & Lab Assistant & Assisted in the generation, collection, and processing of data from experiments. \newline Dealt with Matlab, Excel, and Python, as well as a variety of lab equipment. \\
        \hline
    \end{tabular}
    \vspace{.3in}

    \begin{Large}
        Education \\
    \end{Large}
    \begin{tabular}{ m{3cm} m{6.45cm} m{8cm} }
        \hline
        Fall 2007 - \newline Spring 2010 & UPenn \newline School of Engineering & Majored in Computer Science. Learned some theory, implemented a CPU, an OS shell, and some sweet buffer overflows. Also a little Chinese. Dropped out due to health reasons. \\
        \hline
        Fall 1993 - \newline Spring 2007 & Chestnut Hill Academy & Standard education from kindergarten through high school. Learned some Java and build a 3D solarium. \\
        \hline
    \end{tabular}
    \vspace{.3in}
    
    \begin{Large}
        Of Note \\
    \end{Large}
    \begin{tabular}{ m{3cm} m{3cm} m{3cm} m{8cm}}
        \hline  
        February 18, 2012 & Hacker Olympics & 2nd in Individual and 3rd in Team & Awesome. Best hackathon ever. Consisted of small, silly algorithmic and social engineering challenges. \\
        \hline
        April 20-21, 2010 & HackNY & 2nd (3rd really) & Great. Built beatmon, a social media traffic surge detector and resource swapper for low bandwidth sites, in python with the help of Raymond Ko. \\
        \hline
        January 2010 & PennApps & Did not place & Was a lot of fun. Worked on the Android mobile client for MobLaundry, a washer/dryer status API, as well as the HTML scraper (ugh) to pull data in for the API. \\
        \hline
    \end{tabular}
\end{document}
